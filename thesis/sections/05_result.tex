\section{実験結果と考察}

\subsection{パラメータの密度依存性に関する分析結果}
前章で述べた通り, 本実験ではタスク枯渇による待機時間を排除し, 常に「未割り当てタスク数=エージェント数」が維持される高負荷な条件下でパラメータ探索を行った. 
図\ref{fig:best_params_sweep}に, 各マップにおいてエージェント数(密度)を段階的に変化させた際に獲得された最適パラメータの推移を示す. 

% ======================================================================
% 【ここに画像を挿入してください】
% ======================================================================
\begin{figure}[htbp]
  \centering
  \includegraphics[width=\linewidth]{images/best_params_sweep.png}
  \caption{異なるマップおよびエージェント密度における最適パラメータ値の比較}
  \label{fig:best_params_sweep}
\end{figure}
% ======================================================================

実験結果より, 制限時間内での累積報酬最大化(スループット最大化)を目指す場合, 各パラメータはエージェント密度の変化に対して以下の異なる挙動を示すことが明らかとなった. 

\subsubsection{密度汎化可能なパラメータ}
一部のパラメータは, エージェント数やマップ構造が変化しても最適値が安定しており, 環境に依存しない「汎化パラメータ」として扱えることが確認された. 

\begin{itemize}
    \item \textbf{assign\_drop\_weight}: 
    全ての実験設定において, このパラメータはほぼ一定の値(0付近)に収束した. これは, 常にタスクが供給され続けるLifelong設定において, タスク割り当て時に「配送地点までの距離」を考慮して選り好みするよりも, 近傍のタスクを迅速に確保し回転率を高める戦略が有効であったためと考えられる. したがって, 本パラメータは学習から除外または固定化が可能である. 
    \item \textbf{drop\_weight, pick\_weight}: 
    これらの移動優先度に関する重みは, エージェント数が増減しても最適値の変動が比較的小さい傾向が見られた. これらはマップのトポロジー(物理的な距離感)に依存する部分が大きく, 密度による影響を受けにくいため, 特定のエージェント数で学習した値を転用(汎化)できる可能性が高い. 
\end{itemize}

\subsubsection{密度特化が必要なパラメータ}
一方で, 以下のパラメータはエージェント数に応じて最適値が大きく変動しており, 環境の混雑状況に応じた「特化(チューニング)」が不可欠であることが判明した. 

\begin{itemize}
    \item \textbf{step\_tolerance}: 
    本手法では連続時間でのノード予約を行うため, このパラメータが定める安全マージンは衝突回避性能に直結する. 
    特に本研究では衝突判定に物理的なユークリッド距離を用いているため, \textbf{ノードに接続するエッジ同士のなす角が鋭角であるほど, エージェント間の物理的距離が縮まりやすく, より大きなマージンが必要となる}傾向が見られた. 
    さらに, 密度が高い環境では衝突リスクとデッドロックのトレードオフがシビアになるため, マップ形状と密度の双方に応じた厳密な最適化が要求される.
    
    \item \textbf{assign\_pick\_weight, goal\_weight}: 
    これらは「タスクへの執着度」や「ゴールへの指向性」を制御する. 常に競合が発生する本実験環境下では, エージェント数が少ない場合はゴールへ直行する戦略(高い goal\_weight)が有効だが, 混雑時には譲り合いが必要となるなど, 最適なバランスが密度によって劇的に変化することが示された. 
    
    \item \textbf{congestion\_weight}: 
    周辺の混雑度合いを考慮するこのパラメータも, マップ形状とエージェント密度の組み合わせに強く依存し, 一貫した傾向が見られなかったため, 汎化は困難である. 
\end{itemize}

\subsection{効率的なパラメータ運用指針の提案}
進化計算による全パラメータの同時探索は探索空間が広く, 計算コストが高い. また, 初期段階で頻繁に衝突が発生すると, 他のヒューリスティックパラメータの評価が適切に行えないという問題がある.
そこで, 以上の分析結果に基づき, 計算コストを抑えつつ安定した学習を実現する「3段階の段階的最適化手法」を提案する. 

\begin{enumerate}
    \item \textbf{第1段階(衝突回避の土台作り)}: 
    まず, エージェント数をランダムに変動させた状態で \textbf{step\_tolerance} のみを単独で最適化する. 
    パラメータ学習の初期段階では衝突によるエピソードの早期終了が多発するため, まずはこのパラメータを粗く調整し, エージェントが衝突せずにタスクを遂行できる「最低限の安全性」を確保する土台作りを行う.

    \item \textbf{第2段階(汎化パラメータの学習)}: 
    第1段階で得られた安全マージンを固定した上で, 環境依存性の低い \textbf{pick\_weight} および \textbf{drop\_weight} を最適化する. これらは密度による変動が少ないため, この段階で値を確定させ固定する.

    \item \textbf{第3段階(密度への特化と微調整)}: 
    最後に, 実際の運用密度(エージェント数)に合わせて, 密度依存性の高い \textbf{assign\_pick\_weight}, \textbf{goal\_weight}, そして \textbf{step\_tolerance} の3つを再最適化する. 
    特に step\_tolerance は第1段階で大まかな値が決まっているが, 密度が高まるとデッドロック回避のために微調整が必要となるため, この段階で最終的なチューニングを行う.
\end{enumerate}

このアプローチにより, 学習の収束を早めるとともに, 環境の変化(マップ構造や混雑度の違い)に対して適応力の高いシステムを効率的に構築することが可能となる.
