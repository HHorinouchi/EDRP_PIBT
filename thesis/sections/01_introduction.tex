\section{はじめに}\label{sec-intro}
\subsection{研究の背景}
複数のエージェントを制御する最も基本的なモデルとして,Multi-Agent Path Finding(MAPF)が挙げられる.古くは,グラフ上の小石の移動\cite{Kornhauser84}によって理論的な定式化が行われ,その後,ゲームAIにおける強調的経路探索\cite{Silver05}として再注目され,実用的な探索アルゴリズムの研究が活発化した.MAPFは,各エージェントに対し,開始地点から目的地まで衝突することなく到達する経路を計画する問題であり,物流・倉庫自動化\cite{Wurman08},航空機の移動計画\cite{Morris16},そして自動運転\cite{Dresner08}に応用されている.MAPF問題は,各エージェントが一つの目的地に到達することを目標とする問題であるため,リアルタイムで新たなタスクが発生する倉庫内の荷物配達のような,継続的な課題を解決することができない.そこで,MAPFを拡張させた,Multi-Agent Pickup and Delivery(MAPD)\cite{Ma17}が注目されている.\par

MAPDにおけるタスクは,荷物の「ピックアップ地点」と「配送地点(デリバリー地点)」の2点で構成される.エージェントは現在の位置からピックアップ地点へ移動し,荷物を積載した後,配送地点へと移動する.特に,タスクが継続的に発生するMAPD環境下においては,エージェント間の衝突を回避しつつ,システム全体の単位時間あたりのタスク処理能力(スループット)を最大化することが求められる.\par

MAPDの解決には,「タスク割り当て(Task Assignment)」と「経路計画(Path Planning)」の2つの部分問題を同時に考慮する必要がある.\cite{Ma17}これらは相互に強く依存しており,タスク割り当ての質が経路計画の難易度を左右する.例えば,特定の領域にタスクが集中するような割り当てを行った場合,エージェントの移動経路が重複し,渋滞やデッドロックのリスクが増大する.さらに,実環境ではタスクがリアルタイムに発生するため,計算時間に制約があり,従来のオフライン最適化手法の適用は困難である.したがって,短い周期で計算可能な,軽量かつリアルタイム性の高いアルゴリズムが必要となる.\par

MAPFやMAPDを解く手法には,Conflict-Base Search(CBS)\cite{Sharon15}やToken Passing(TP)\cite{Ma17},Priority Inheritance with Backtracking(PIBT)\cite{Okumura19}が挙げられる.CBSは,下位エージェントが貪欲に経路を探索し,上位エージェントが下位エージェントの衝突を検知し,当該衝突を解消するための制約を課すことで衝突のない経路を探索する.TPは,最優先に経路を計画するエージェントを一つ選択し,それ以外のエージェントが最優先エージェントとの衝突を避けるように経路を探索する.PIBTは,各エージェントの次の1ステップの行動を,優先度に基づいて逐次的に決定するアルゴリズムであり,計算コストが低くリアルタイム環境との親和性が高い.本研究では,多くの研究が想定しているグリッド環境ではなく,状態空間が大きな無向グラフ空間上での性能を比較することから,計算コストの低さとに着目し,PIBTを応用する形で提案手法に用いる.
\par
PIBTの性能は,競合発生時の移動順序を決定する「エージェントの優先度」の定義に大きく依存する.しかし,マップ構造やエージェント密度などの環境要因に応じて最適な優先度規則を手動で設計することは困難である.そこで本研究では,進化戦略(ES: Evolution Strategy)を用いた機械学習アプローチにより,環境に適した優先度決定パラメータを自動獲得する手法を検討する.\par

実運用を想定した場合,導入されるマップ構造は固定的である一方で,稼働するエージェント数やタスク発生頻度は変動する可能性がある.そのため,特定の環境に特化して過学習させたパラメータと,多様な環境で学習させた汎化的なパラメータの間には,性能および安定性においてトレードオフが存在すると考えられる.本研究では,特定のマップ構造に特化させた場合と,未知の環境への適応を考慮した場合の双方においてパラメータの学習を行い,その性能差と有効性を検証する.


\subsection{研究の目的}
本研究の目的は,"特定のマップ構造に特化してPIBTの優先度パラメータを学習させることで,汎用的に学習させた場合よりも高い性能(スループット)が得られる"という仮説を定量的に検証することである.\par
具体的には,特定の環境に過適合(Overfitting)させた「特化型モデル」と,多様な環境で学習させた「汎化型モデル」を構築し,それぞれのタスク処理能力や衝突回避性能を比較評価する.得られた実験結果に基づき,マップのトポロジー(構造的特徴)と最適パラメータの関係性を分析することで,特定の物流現場において特化型学習を適用することの実用上の有効性を示す.さらに,獲得されたパラメータ群の中で,どの要素が性能差に最も大きく寄与しているかを考察し,効率的な運用に向けた指針を明らかにする.

\subsection{本論文の構成}
本論文の構成は以下の通りである.\\
まず2章では本研究の背景となる研究,および関連研究について述べる.3章では実際に用いるアルゴリズムや学習手法を含めた実験の方法を述べる.4章では実験結果について記し,5章では結果についての考察を述べる.6章では,結論と今後の展望について記す.
