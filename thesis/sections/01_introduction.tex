\section{はじめに}\label{sec-intro}
\subsection{研究の背景}
複数のエージェントを制御する最も基本的なモデルとして, Multi-Agent Path Finding (MAPF) が挙げられる. 古くは, グラフ上の小石の移動\cite{Kornhauser84}によって理論的な定式化が行われ, その後, ゲームAIにおける協調的経路探索\cite{Silver05}として再注目され, 実用的な探索アルゴリズムの研究が活発化した. MAPFは, 各エージェントに対し, 開始地点から目的地まで衝突することなく到達する経路を計画する問題であり, 物流・倉庫自動化\cite{Wurman08}, 航空機の移動計画\cite{Morris16}, そして自動運転\cite{Dresner08}に応用されている. 最も基本的なMAPF問題は, 各エージェントが一つの目的地に到達することを目標とする問題であるため, リアルタイムで新たなタスクが発生する倉庫内の荷物配達のような, 継続的な課題を解決することができない. そこで, MAPFを拡張させた, Multi-Agent Pickup and Delivery (MAPD)\cite{Ma17}が提案された. \par

MAPDにおけるタスクは, 荷物の「ピックアップ地点」と「配送地点(デリバリー地点)」の2点で構成される. エージェントは現在の位置からピックアップ地点へ移動し, 荷物を積載した後, 配送地点へと移動する. 特に, タスクが継続的に発生するMAPD環境下においては, エージェント間の衝突を回避しつつ, システム全体の単位時間あたりのタスク処理能力(スループット)を最大化することが求められる. \par

MAPDの解決には, 「タスク割り当て (Task Assignment) 」と「経路計画 (Path Planning) 」の2つの部分問題を同時に考慮する必要がある\cite{Ma17}. これらは相互に強く依存しており, タスク割り当ての質が経路計画の難易度を左右する. 例えば, 特定の領域にタスクが集中するような割り当てを行った場合, エージェントの移動経路が重複し, 渋滞やデッドロックのリスクが増大する. さらに, 実環境ではタスクがリアルタイムに発生するため, 計算時間に制約があり, 従来のオフライン最適化手法の適用は困難である. したがって, 短い周期で計算可能な, 軽量かつリアルタイム性の高いアルゴリズムが必要となる. \par

MAPFやMAPDを解く手法には, Conflict-Based Search (CBS) \cite{Sharon15}やToken Passing (TP)\cite{Ma17}, Priority Inheritance with Backtracking (PIBT) \cite{Okumura19}が挙げられる. CBSは, 下位エージェントが貪欲に経路を探索し, 上位エージェントが下位エージェントの衝突を検知し, 当該衝突を解消するための制約を課すことで衝突のない経路を探索する. TPは, 最優先に経路を計画するエージェントを一つ選択し, それ以外のエージェントが最優先エージェントとの衝突を避けるように経路を探索する. PIBTは, 各エージェントの次の1ステップの行動を, 優先度に基づいて逐次的に決定するアルゴリズムであり, 計算コストが低くリアルタイム環境との親和性が高い. 本研究では, 多くの研究が想定しているグリッド環境ではなく, 状態空間が大きな無向グラフ空間上での性能を比較することから, 計算コストの低さに着目し, PIBTをベースにグラフ環境および連続空間へと拡張した \textbf{Graph-based Adaptive Parameter PIBT (GAP-PIBT)} を提案する. 
\par
GAP-PIBT の基盤となる PIBT の性能は, 競合発生時の移動順序を決定する「エージェントの優先度」の定義に大きく依存する. しかし, マップ構造やエージェント密度などの環境要因に応じて最適な優先度規則を手動で設計することは困難である. そこで本研究の提案する GAP-PIBT では, 進化戦略 (ES: Evolution Strategy) を用いた機械学習アプローチを導入し, 環境に適した優先度決定パラメータを自動獲得するフレームワークを構築する. \par

実運用を想定した場合, 導入されるマップ構造は固定的である一方で, 稼働するエージェント数やタスク発生頻度は変動する可能性がある. そのため, 特定の環境に特化して過学習させたパラメータと, 多様な環境で学習させた汎化的なパラメータの間には, 性能および安定性においてトレードオフが存在すると考えられる. 本研究では, 特定のマップ構造に特化させた場合と, 未知の環境への適応を考慮した場合の双方において GAP-PIBT の学習を行い, その性能差と有効性を検証する. 


\subsection{研究の目的} 本研究の主目的は, 特定の環境構造にパラメータを適合させる「特化 (Specialization)」と, 多様な環境に対応可能な「汎化 (Generalization)」の間に存在するトレードオフ関係を定量的に解明することである. 一般に機械学習において特定の環境への過剰適合 (Overfitting) は忌避されるが, 本研究ではこれを特定の物流現場における「性能上限の追求」と捉え直し, 汎用的なパラメータ運用との性能乖離 (Gap) を分析する. \par

具体的には, 以下の3点を達成することを目的とする.  \begin{enumerate} \item \textbf{特化と汎化の性能差の定量化}: 単一環境に特化させたモデルと, 複数環境で学習させた汎化モデルを構築し, 両者のスループットおよび衝突回避性能の差異を測定することで, 汎用性を確保するために生じる「性能コスト」を明らかにする.  \item \textbf{環境構造と最適パラメータの相関分析}: 獲得された特化パラメータがマップのトポロジー (ノード密度やボトルネック構造) によってどのように変化するかを分析し, 環境特性に応じた最適な優先度設計の法則性を導出する.  \item \textbf{実運用に向けた指針の提示}: 上記の分析に基づき, 実環境への導入において「コストをかけて環境ごとに再学習を行うべきか, あるいは汎用パラメータで十分か」を判断するための定量的な指針を提示する.  \end{enumerate}

\subsection{本論文の構成}
本論文の構成は以下の通りである. \\
まず2章では本研究の背景となる研究, および関連研究について述べる. 3章では提案手法 GAP-PIBT の詳細なアルゴリズムや学習手法を含めた実験の方法を述べる. 4章では実験結果について記し, 5章では, 結論と今後の展望について記す. 
