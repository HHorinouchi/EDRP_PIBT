\documentclass{kuisthesis}           % 日本語 
%\documentclass[english]{kuisthesis} % 英語

\jtitle[PIBT-G における優先度パラメータの\\
特定マップ最適化と汎化性能のトレードオフ分析]%	% 和文題目(内容梗概/目次用)
	{PIBT-G における優先度パラメータの
特定マップ最適化と汎化性能のトレードオフ分析}	% 和文題目
\etitle{Trade-off Analysis Between Specific Map Optimization and Generalization Performance of Priority Parameters in PIBT-G}	% 英文題目
\jauthor{堀之内 駿輝}				% 和文著者名
\eauthor{Hayaki Horinouchi}			% 英文著者名
\supervisor{伊藤孝之 教授}			% 指導教員名
\date{\today}				% 提出年月日

\begin{document}
\maketitle					% 「とびら」の出力


% ------- 論文要旨 -------
% 独立ページ扱いなので、ファイル内で\clearpageを入れても良いですが、
% ここで区切ったほうが構造が明確です。
\begin{jabstract}				% 和文梗概
近年,電子商取引の拡大や労働力不足を背景に,物流倉庫やドローン配送などの実世界アプリケーションにおいて,複数の自律移動ロボットを用いた搬送作業の自動化が急速に進展している.このようなシステムでは,複数のエージェントが衝突することなく,かつ効率的にタスクを処理するための経路計画,すなわち Multi-Agent Pickup and Delivery (MAPD) 問題の解決が不可欠である.従来の MAPD 研究の多くは,環境を均質なグリッド世界としてモデル化し,離散的な時間・空間上でのアルゴリズム検証を行ってきた.しかし,実際のドローン運航や複雑なレイアウトを持つ倉庫環境などを考慮すると,移動空間は必ずしもグリッド構造には限定されず,ノード間の距離やエージェントの物理的な速度,安全距離といった連続的な制約を含む一般的な無向グラフとしてモデル化することが,実用的な観点から極めて重要となる.\par

本研究では,このような物理的制約を考慮したグラフ環境上での Lifelong MAPD 問題を対象とし,分散型経路計画アルゴリズムである Priority Inheritance with Backtracking (PIBT) を用いた効率的な配送制御手法について論じる.PIBT は計算コストが低くリアルタイム性に優れる一方で,その探索性能やデッドロック回避能力は,各エージェントの行動優先度を決定するヒューリスティックなパラメータに強く依存するという性質を持つ.従来,この優先度は目的地までの距離など単純な指標で決定されることが多かったが,複雑なグラフ構造や変動するタスク密度においては,混雑度やタスクの進行状態(ピックアップ前後)を考慮した動的な優先度調整が必要となる.しかし,これらのパラメータを環境ごとに人手で調整することは困難であり,環境の特性に応じた最適なパラメータ設計が課題となっている.\par

そこで本研究では,進化計算の一種である Evolution Strategy (ES) を用いて,PIBT の優先度決定パラメータおよびタスク割り当てパラメータを自動最適化するフレームワークを提案する.具体的には,目的地への距離,待機ペナルティ,周辺の混雑度,およびタスクの処理段階に応じた重み付け係数を遺伝子として定義し,シミュレーション環境におけるタスク処理数や衝突率,移動コストを報酬関数として学習を行う.提案手法の特徴は,単に特定の環境で性能を最大化する「特化型」の学習だけでなく,異なるマップ構造やエージェント数においても一定の性能を維持する「汎化性能」に着目している点にある.特定のマップ(例:渋谷マップなど)やエージェント密度に過剰適合させたパラメータは,その環境下では極めて高いスループットを発揮する一方で,未知の環境に対しては脆弱になる可能性がある.本研究では,単一環境学習と複数環境学習の比較を通じて,環境構造と最適な優先度パラメータの相関関係を分析する.

\end{jabstract}

\begin{eabstract}				% 英文梗概
This guide describes how to write your guraduation thesis
according to the regulation of Computer Science Course, 
School of Informatics and Mathematical Science,
Faculty of Engineering Kyoto University. 
This regulation specifies the rules about the structure 
and format of the thesis which you need to follow in writing.
This guide also explains how to use a \LaTeX{} style file for
graduation thesis, named \verb|kuisthesis|, 
with which you can easily produce a well-formatted thesis. 
This guide itself is written using \verb|kuisthesis|; 
the source code may be helpful 
if you would like to know how to use this style file.
\end{eabstract}

\clearpage

% ------- 目次 -------
\tableofcontents

% 各章の読み込み
% \input コマンドは、指定したファイルの中身をそのままここに展開します。

\section{はじめに}\label{sec-intro}
\subsection{研究の背景}
複数のエージェントを制御する最も基本的なモデルとして, Multi-Agent Path Finding (MAPF) が挙げられる. 古くは, グラフ上の小石の移動\cite{Kornhauser84}によって理論的な定式化が行われ, その後, ゲームAIにおける協調的経路探索\cite{Silver05}として再注目され, 実用的な探索アルゴリズムの研究が活発化した. MAPFは, 各エージェントに対し, 開始地点から目的地まで衝突することなく到達する経路を計画する問題であり, 物流・倉庫自動化\cite{Wurman08}, 航空機の移動計画\cite{Morris16}, そして自動運転\cite{Dresner08}に応用されている. 最も基本的なMAPF問題は, 各エージェントが一つの目的地に到達することを目標とする問題であるため, リアルタイムで新たなタスクが発生する倉庫内の荷物配達のような, 継続的な課題を解決することができない. そこで, MAPFを拡張させた, Multi-Agent Pickup and Delivery (MAPD)\cite{Ma17}が提案された. \par

MAPDにおけるタスクは, 荷物の「ピックアップ地点」と「配送地点(デリバリー地点)」の2点で構成される. エージェントは現在の位置からピックアップ地点へ移動し, 荷物を積載した後, 配送地点へと移動する. 特に, タスクが継続的に発生するMAPD環境下においては, エージェント間の衝突を回避しつつ, システム全体の単位時間あたりのタスク処理能力(スループット)を最大化することが求められる. \par

MAPDの解決には, 「タスク割り当て (Task Assignment) 」と「経路計画 (Path Planning) 」の2つの部分問題を同時に考慮する必要がある\cite{Ma17}. これらは相互に強く依存しており, タスク割り当ての質が経路計画の難易度を左右する. 例えば, 特定の領域にタスクが集中するような割り当てを行った場合, エージェントの移動経路が重複し, 渋滞やデッドロックのリスクが増大する. さらに, 実環境ではタスクがリアルタイムに発生するため, 計算時間に制約があり, 従来のオフライン最適化手法の適用は困難である. したがって, 短い周期で計算可能な, 軽量かつリアルタイム性の高いアルゴリズムが必要となる. \par

MAPFやMAPDを解く手法には, Conflict-Based Search (CBS) \cite{Sharon15}やToken Passing (TP)\cite{Ma17}, Priority Inheritance with Backtracking (PIBT) \cite{Okumura19}が挙げられる. CBSは, 下位エージェントが貪欲に経路を探索し, 上位エージェントが下位エージェントの衝突を検知し, 当該衝突を解消するための制約を課すことで衝突のない経路を探索する. TPは, 最優先に経路を計画するエージェントを一つ選択し, それ以外のエージェントが最優先エージェントとの衝突を避けるように経路を探索する. PIBTは, 各エージェントの次の1ステップの行動を, 優先度に基づいて逐次的に決定するアルゴリズムであり, 計算コストが低くリアルタイム環境との親和性が高い. 本研究では, 多くの研究が想定しているグリッド環境ではなく, 状態空間が大きな無向グラフ空間上での性能を比較することから, 計算コストの低さに着目し, PIBTをベースにグラフ環境および連続空間へと拡張した \textbf{Graph-based Adaptive Parameter PIBT (GAP-PIBT)} を提案する. 
\par
GAP-PIBT の基盤となる PIBT の性能は, 競合発生時の移動順序を決定する「エージェントの優先度」の定義に大きく依存する. しかし, マップ構造やエージェント密度などの環境要因に応じて最適な優先度規則を手動で設計することは困難である. そこで本研究の提案する GAP-PIBT では, 進化戦略 (ES: Evolution Strategy) を用いた機械学習アプローチを導入し, 環境に適した優先度決定パラメータを自動獲得するフレームワークを構築する. \par

実運用を想定した場合, 導入されるマップ構造は固定的である一方で, 稼働するエージェント数やタスク発生頻度は変動する可能性がある. そのため, 特定の環境に特化して過学習させたパラメータと, 多様な環境で学習させた汎化的なパラメータの間には, 性能および安定性においてトレードオフが存在すると考えられる. 本研究では, 特定のマップ構造に特化させた場合と, 未知の環境への適応を考慮した場合の双方において GAP-PIBT の学習を行い, その性能差と有効性を検証する. 


\subsection{研究の目的} 本研究の主目的は, 特定の環境構造にパラメータを適合させる「特化 (Specialization)」と, 多様な環境に対応可能な「汎化 (Generalization)」の間に存在するトレードオフ関係を定量的に解明することである. 一般に機械学習において特定の環境への過剰適合 (Overfitting) は忌避されるが, 本研究ではこれを特定の物流現場における「性能上限の追求」と捉え直し, 汎用的なパラメータ運用との性能乖離 (Gap) を分析する. \par

具体的には, 以下の3点を達成することを目的とする.  \begin{enumerate} \item \textbf{特化と汎化の性能差の定量化}: 単一環境に特化させたモデルと, 複数環境で学習させた汎化モデルを構築し, 両者のスループットおよび衝突回避性能の差異を測定することで, 汎用性を確保するために生じる「性能コスト」を明らかにする.  \item \textbf{環境構造と最適パラメータの相関分析}: 獲得された特化パラメータがマップのトポロジー (ノード密度やボトルネック構造) によってどのように変化するかを分析し, 環境特性に応じた最適な優先度設計の法則性を導出する.  \item \textbf{実運用に向けた指針の提示}: 上記の分析に基づき, 実環境への導入において「コストをかけて環境ごとに再学習を行うべきか, あるいは汎用パラメータで十分か」を判断するための定量的な指針を提示する.  \end{enumerate}

\subsection{本論文の構成}
本論文の構成は以下の通りである. \\
まず2章では本研究の背景となる研究, および関連研究について述べる. 3章では提案手法 GAP-PIBT の詳細なアルゴリズムや学習手法を含めた実験の方法を述べる. 4章では実験結果について記し, 5章では, 結論と今後の展望について記す. 

\section{背景となる研究}\label{sec-background}
本章では, 本研究の基礎となるMulti-Agent Path Finding (MAPF) 問題, およびその応用問題であるMulti-Agent Pickup and Delivery (MAPD) 問題について概説する. 特に, 本研究ではドローン配送などの実世界応用を念頭に置き, MAPFを物理的制約を考慮して拡張したモデルであるDrone Routing Problem (DRP) \cite{Kaji2025} の定義に基づくグラフベースのモデルを採用する. 続いて, 本研究で採用する分散型アルゴリズムであるPIBTについて述べ, その課題を明らかにする. 最後に, PIBTにおけるパラメータ最適化手法として採用する進化戦略 (Evolution Strategy) について解説する. 

% =============================================
% 2.1 MAPFとDRP
% =============================================
\subsection{MAPFからの拡張モデル: DRP}
Multi-Agent Path Finding (MAPF) 問題は, 複数のエージェントが現在位置から目的地まで, 互いに衝突することのない経路を計画する問題である. すべてのエージェントが目的地に到達するまでのコスト(総移動時間や移動距離の和など)を最小化することを目標とする. \par

従来の基本的なMAPF問題は, 図\ref{fig:two_images}(a)に示すようなグリッド環境上で定義され, エージェントは離散的なセル間を移動するモデルが一般的であった. しかしながら, ドローン配送のような実空間での運用を考慮した場合, 移動空間は必ずしも均質なグリッド構造には限定されない. Kajiらは, このような問題をDrone Routing Problem (DRP)\cite{Kaji2025}として定義し, ノード間の物理的な距離やエージェントの連続的な位置を考慮した2次元グラフマップ上での経路計画問題として定式化した. 
そこで本研究では, 環境をグリッドではなく図\ref{fig:two_images}(b)のような一般的な無向グラフとして定義し, 従来のMAPF問題を物理的な移動制約および連続空間での衝突回避を考慮して拡張したDRPモデルを取り扱う. 

\begin{figure}[htbp]
  \centering
  % --- 左の画像 ---
  \begin{subfigure}[b]{0.45\textwidth}
    \centering
    % 画像ファイルがある場合
    \includegraphics[width=\linewidth]{images/grid_map.png}
    
    % 手動でキャプションを書く
    \centerline{(a) グリッド環境のマップ}
  \end{subfigure}
  \hfill % 画像間のスペース調整
  % --- 右の画像 ---
  \begin{subfigure}[b]{0.45\textwidth}
    \centering
    \includegraphics[width=\linewidth]{images/graph_map.png}
    
    % 手動でキャプションを書く
    \centerline{(b) 無向グラフ環境のマップ (DRPモデル)}
  \end{subfigure}
  % 全体のキャプション
  \caption{2つの環境モデルの比較}
  \label{fig:two_images} % ← 親のラベルは有効
\end{figure}

\subsubsection{問題の定義}
本研究では, Kajiら\cite{Kaji2025}の定義に基づき, DRPを物理的制約を持つグラフ上でのMAPF問題として定式化する. 
問題はタプル $\mathcal{M} = \langle G,  \mathcal{A},  \mathcal{T},  \Sigma,  \mathcal{C},  \mathcal{W} \rangle$ によって定義される. 

\begin{itemize}
    \item $G = (V,  E)$ は無向グラフであり, $V$ はノード集合, $E \subseteq V \times V$ はエッジ集合を表す. 
    \item $\mathcal{A} = \{a_1,  a_2,  \dots,  a_m\}$ は $m$ 体のエージェント(ドローン)の集合である. 
    \item $\mathcal{T} = \{0,  1,  2,  \dots \}$ は離散化された時刻(タイムステップ)の集合である. 
    \item $\Sigma = \{(s_i,  g_i) \mid s_i,  g_i \in V,  a_i \in \mathcal{A}\}$ は各エージェントの始点ノード $s_i$ と終点ノード $g_i$ のペアの集合である. 
    \item $\mathcal{C}: V \to \mathbb{R}^2$ は, 各ノード $v \in V$ を2次元ユークリッド空間上の座標に対応付ける関数である. 
    \item $\mathcal{W}: E \to \mathbb{R}^+$ は各エッジ $(u,  v) \in E$ の物理的な長さ(距離)を定義する重み関数であり, $\mathcal{W}(u,  v) = \lVert \mathcal{C}(u) - \mathcal{C}(v) \rVert$ で与えられる. 
\end{itemize}

さらに, 各エージェント $a_i$ には以下の物理パラメータが与えられる. 
\begin{itemize}
    \item $v_{\text{max}} \in \mathbb{R}^+$: エージェントが1タイムステップに進むことができる最大物理距離(速度). 
    \item $r_{\text{safe}} \in \mathbb{R}^+$: 衝突回避のために維持すべき安全半径(エージェント間の最小距離は $2r_{\text{safe}}$). 
\end{itemize}

\subsubsection{状態と行動モデル}
各タイムステップ $t \in \mathcal{T}$ におけるエージェント $a_i$ の状態は, 物理的な位置座標 $\bm{p}_i(t) \in \mathbb{R}^2$ および, グラフ上の所在(直前のノード $u$, 向かっているノード $v$)によって管理される. 
初期状態は $\bm{p}_i(0) = \mathcal{C}(s_i)$ である. 

エージェントは各ステップにおいて, 隣接ノードへの移動または待機を選択する. 
行動 $act_{i, t}$ がノード $v_{next} \in V$ への移動である場合, エージェントは現在の位置から $v_{next}$ に向かって最大 $v_{\text{max}}$ だけ進む. 
位置座標 $\bm{p}_i(t)$ の更新は以下のように定義される. 

\begin{equation}
    \bm{p}_i(t+1) = \bm{p}_i(t) + \delta \cdot \bm{u}
\end{equation}

ここで, $\bm{u}$ は進行方向の単位ベクトル, $\delta$ は実際の移動距離を表す. 現在のターゲットノードを $v_{next}$ とすると, これらは以下のように計算される. 

\begin{align}
    \bm{d} &= \mathcal{C}(v_{next}) - \bm{p}_i(t) \\
    \bm{u} &= \frac{\bm{d}}{\lVert \bm{d} \rVert} \\
    \delta &= \min(v_{\text{max}},  \lVert \bm{d} \rVert)
\end{align}

すなわち, DRPにおけるエージェントは, 離散的な時間ステップごとにグラフのエッジ上を連続的に遷移する. 目的地までの距離が $v_{\text{max}}$ 未満の場合は, オーバーシュートすることなくノード位置 $\mathcal{C}(v_{next})$ に正確に停止する. 

\subsubsection{制約条件と目的関数}
有効な解(行動計画)は, 以下の衝突回避制約を常に満たす必要がある. 

\begin{equation}
    \forall t \in \mathcal{T},  \forall a_i,  a_j \in \mathcal{A} \ (i \neq j),  \quad \lVert \bm{p}_i(t) - \bm{p}_j(t) \rVert \geq 2r_{\text{safe}}
\end{equation}

本研究では, ノード上での待機中およびエッジ移動中の双方において, この物理的距離制約を厳密に適用する. 
問題の目的は, 全エージェントが衝突制約を満たしながら始点 $s_i$ から終点 $g_i$ へ到達するまでの総所要時間(Makespan)または総移動コスト(Total Travel Time)を最小化することである. 

% =============================================
% 2.2 MAPD
% =============================================
\subsection{Multi-Agent Pickup and Delivery (MAPD)}
Multi-Agent Pickup and Delivery (MAPD) 問題は, MAPF問題に対してタスクの概念を拡張し, 各タスクに「ピックアップ地点」と「デリバリー地点」を定義した問題である. 
従来のMAPFが単純に始点から終点への移動を扱うのに対し, MAPDではエージェントがピックアップ地点で荷物を積載し, デリバリー地点まで運搬するという一連のプロセスを計画する必要がある. \par
さらに, 本研究で扱う \textbf{Lifelong MAPD} の設定では, タスクが時間の経過とともに動的かつ継続的に発生するため, エージェントは一度の配送で終了することなく, 新たなタスクを次々と処理し続けることが求められる. 

\subsubsection{タスクとシステム構成}
本研究におけるMAPD環境は, 前節で定義したMAPFのタプル $\mathcal{M}$ に加え, タスクセット $\Theta$ を導入することで定義される. 
タスクセット $\Theta = \{ \tau_1,  \tau_2,  \dots \}$ は, システムに投入される配送リクエストの集合である. 各タスク $\tau_j \in \Theta$ は以下のタプルで表される. 

\begin{equation}
    \tau_j = \langle v_{j}^{\text{pick}},  v_{j}^{\text{drop}},  t_{j}^{\text{release}} \rangle
\end{equation}

ここで, 各要素は以下の通りである. 
\begin{itemize}
    \item $v_{j}^{\text{pick}} \in V$: 荷物のピックアップ場所(始点). 
    \item $v_{j}^{\text{drop}} \in V$: 荷物の配送場所(終点). 
    \item $t_{j}^{\text{release}} \in \mathcal{T}$: タスクがシステムに発生し, 割り当て可能となる時刻. 
\end{itemize}

\subsubsection{エージェントの状態遷移}
MAPDにおいて, 各エージェント $a_i \in \mathcal{A}$ は, 自身のタスク保持状況に応じて以下の3つの状態のいずれかをとる. 

\begin{enumerate}
    \item \textbf{Free (待機状態)}: タスクを割り当てられていない状態. 
    \item \textbf{ToPick (ピックアップ移動中)}: タスク $\tau_j$ を割り当てられ, 現在位置から $v_{j}^{\text{pick}}$ へ移動している状態. 
    \item \textbf{ToDrop (配送移動中)}: $v_{j}^{\text{pick}}$ に到達して荷物を積載し, 目的地 $v_{j}^{\text{drop}}$ へ移動している状態. 
\end{enumerate}

\subsubsection{MAPFとの関係}
MAPDは, タスク割り当て層と経路計画層の2層構造として捉えることができる. 
各時刻 $t$ において, Free状態のエージェントに対し未割り当てのタスク $\tau_j$ が割り当てられると, そのエージェントの目的地 $g_i$ が更新される. 

\begin{equation}
    g_i = 
    \begin{cases}
        v_{j}^{\text{pick}} & (\text{if state is ToPick}) \\
        v_{j}^{\text{drop}} & (\text{if state is ToDrop}) \\
        \text{nil} & (\text{if state is Free})
    \end{cases}
\end{equation}

このようにして動的に決定される目的地 $g_i$ と, 現在のエージェント位置 $\mathbf{p}_i(t)$ を入力として, 前節で定義したMAPF問題を各タイムステップ(または一定間隔)で解くことにより, エージェントの具体的な行動 $act_{i, t}$ が決定される. 
本研究の目的は, 全てのタスクを完了させるまでの総所要時間, あるいは単位時間あたりのタスク処理数(スループット)を最適化することである. 

% =============================================
% 2. 3 PIBT
% =============================================
\subsection{Priority Inheritance with Backtracking (PIBT)}
Priority Inheritance with Backtracking (PIBT) \cite{Okumura19} は, 各タイムステップにおいてエージェントの次の移動先を決定するための分散型・反応型のアルゴリズムである. 
従来のCBSのような探索ベースの手法が「全エージェントの全経路」を事前に計算するのに対し, PIBTは「現在の1ステップ先の移動」のみを非常に低い計算コストで決定する. そのため, エージェント数が多い大規模環境や, タスクが逐次発生するMAPDのようなオンライン環境において優れたスケーラビリティを発揮する. 
なお, 本節では説明の簡略化のため, 図\ref{fig:two_images}(a)のようなグリッド環境を仮定してPIBTのロジックを解説する. 

\subsubsection{動作原理}
PIBTによる衝突回避は, 主に以下の3つのメカニズムによって実現される. 

\begin{description}
  \item[優先度に基づく意思決定]
  各タイムステップにおいて, 全てのエージェントには何らかの規則に基づいて「優先度」が付与される. PIBTは, 優先度の高いエージェントから順に次の移動先を決定する. 優先度が高いエージェント $a_i$ が隣接ノード $v$ へ移動しようとした際, そのノードが空いている場合は即座に移動が確定する. 

  \item[優先度継承 (Priority Inheritance)]
  PIBTの最大の特徴である. 高優先度のエージェント $a_i$ が移動したいノード $v$ に, 未だ移動先が決定していない低優先度のエージェント $a_j$ が存在する場合, $a_i$ は $a_j$ に対して自身の優先度を一時的に「継承」させる. 
  優先度を受け取った $a_j$ は, 本来の自身の順番を待つことなく, 直ちに $a_i$ のために場所を空けるよう移動先を探索・決定する. これにより, 高優先度のエージェントの進行方向にある障害(他エージェント)が玉突き的に排除され, スムーズな移動が可能となる. 

  \item[バックトラッキング (Backtracking)]
  もし $a_j$ が周囲を他の高優先度エージェントや障害物に囲まれており, 有効な移動先が見つからない場合, 移動の試行は失敗とみなされ, 呼び出し元である $a_i$ に失敗が通知される. これを受け, $a_i$ は別の候補ノードへの移動を試みるか, その場で待機することになる. このプロセスは再帰的に行われる. 
\end{description}

\subsubsection{アルゴリズムの手順}
上述の原理に基づく具体的な手続きは以下の通りである. 各タイムステップ $t$ において, システムは以下の処理を実行する. 

\begin{enumerate}
  \item 全エージェントを優先度の降順にソートする. 
  \item 未計画のエージェントの中で最も優先度の高いものから順に, 関数 \texttt{move($a_i$)} を呼び出す. 
  \item \texttt{move($a_i$)} の内部処理:
  \begin{itemize}
    \item $a_i$ の目的地への距離が短くなるような隣接ノード $v$ を候補とする. 
    \item $v$ が空いている (占有予定がない) 場合, その場所を確保し終了する. 
    \item $v$ に他のエージェント $a_j$ がいる場合:
    \begin{itemize}
        \item もし $a_j$ の移動先が既に決定済みであれば, $v$ への移動は不可とし, 別の候補を探す. 
        \item もし $a_j$ の移動先が未定であれば, $a_j$ に優先度を継承し, 再帰的に \texttt{move($a_j$)} を呼び出す. 
        \item $a_j$ が移動に成功すれば, $a_i$ は $v$ を確保する. 失敗すれば $a_i$ は $v$ への移動を諦め, 次の候補を探すか待機を選択する. 
    \end{itemize}
  \end{itemize}
\end{enumerate}

このアルゴリズムにより, 局所的な衝突を回避しながら, 計算時間をエージェント数に対して線形に近いオーダーに抑えることができる. しかし, PIBTの性能 (スループットや到達率) は「どのエージェントに高い優先度を与えるか」という優先度規則に強く依存する. 単純な規則 (例:ランダム, ID順) では, 複雑なマップ構造においてデッドロックや非効率な動きが発生しやすいため, 本研究ではこの優先度決定の最適化に焦点を当てる. 


% =============================================
% 2.4 進化戦略 (Evolution Strategy)
% =============================================
\subsection{進化戦略 (Evolution Strategy)}
本研究では, PIBTにおけるヒューリスティックな優先度パラメータを最適化するために, 進化戦略 (Evolution Strategy: ES) を用いる. 進化戦略は, 生物の進化の過程に着想を得たブラックボックス最適化手法の一種であり, 関数の勾配情報が得られない, あるいは勾配の計算が困難な問題に対して有効である\cite{Rechenberg73}. 

強化学習 (Reinforcement Learning: RL) がエージェントの行動に対する価値関数や方策勾配を学習するのに対し, 進化戦略はパラメータ空間上で直接探索を行う. Salimansらは, 進化戦略が深層強化学習の代替として競争力のある性能を発揮し, 特に並列化効率において優れていることを示した\cite{Salimans17}. MAPD問題のようなマルチエージェントシミュレーションは, エージェント間の相互作用が複雑であり, 報酬関数が微分不可能であるため, 勾配を用いないESのアプローチは適している. 

% 微分不可能性について
% エージェントの行動は「右に移動」「待機」といった離散的な選択です.  プログラム内部では argmax(最大値を選ぶ操作)や if 文(条件分岐)が多用されます. 

% 例: if 距離 < 安全距離 then 停止 else 移動 このような「条件分岐」や「整数の座標移動」は, 入力パラメータを 0. 0001 だけ変えても出力(行動)が全く変わらない(勾配が 0)か, ある閾値で急激に変わる(勾配が無限大/不連続)ため, 数学的な微分ができません. 

\subsubsection{アルゴリズムの概要}
一般的な進化戦略(Natural Evolution Strategiesの変種)では, パラメータベクトル $\theta$ を中心とした正規分布 $\mathcal{N}(\theta,  \sigma^2 I)$ から, 摂動(ノイズ) $\epsilon_i$ を加えた $N$ 個の候補パラメータ $\theta_i = \theta + \sigma \epsilon_i$ を生成する. ここで $\sigma$ はノイズの標準偏差である. 
各候補パラメータを用いて環境でシミュレーションを行い, 得られた報酬 $F(\theta_i)$ を評価値とする. パラメータの更新は, 高い報酬を得た方向へ分布の中心を移動させることで行われる. 更新則は次式で近似される. 

\begin{equation}
    \theta \leftarrow \theta + \alpha \frac{1}{N \sigma} \sum_{i=1}^{N} F(\theta + \sigma \epsilon_i) \epsilon_i
\end{equation}

ここで, $\alpha$ は学習率を表す. この更新式は, 期待報酬の勾配を有限差分法により確率的に推定していると解釈できる. 

\subsubsection{安定化のための技術}
本研究の実装においては, 学習の安定性と収束速度を向上させるために, 以下の2つの主要な技術を導入している. 

\begin{description}
    \item[対抗変数法 (Antithetic Sampling)]
    勾配推定の分散を低減させるための手法である\cite{Geweke88}. ランダムなノイズ $\epsilon$ を生成する際, $\epsilon$ とその反転である $-\epsilon$ のペアを同時に個体群として生成する. これにより, 正負の摂動に対する対称的なサンプリングが保証され, サンプリングノイズによる推定精度の悪化を抑制する効果がある. 

    \item[適合度シェイピング (Fitness Shaping)]
    得られた報酬 $F(\theta_i)$ をそのまま重みとして使うのではなく, 集団内での順位 (ランク) に基づいて変換する手法である\cite{Wierstra14}. 具体的には, 報酬の値を昇順に並べ替え, [-0.5,  0.5] の範囲などに正規化して利用する.これにより, 極端に大きな報酬値 (外れ値) が勾配推定に与える過度な影響を排除し, 局所解への早期収束を防ぐロバストな学習が可能となる.
\end{description}

本研究では, これらの拡張を施した進化戦略を用いて, PIBTにおける優先度決定のための重みパラメータ (ゴールへの距離, エージェント間の混雑度などの重み係数) を最適化する.

\section{アプローチ}\label{sec-approach}

\section{実験}\label{sec-experiment}

\section{実験結果}\label{sec-result}

\section{結論}\label{sec-couclusion}
\subsection{本研究のまとめ}
本研究では, [研究の背景・目的] に対し, [提案手法名] を提案し, その有効性を検証した.
具体的には, 以下の取り組みを行った.

\begin{itemize}
    \item \textbf{提案手法の構築}:
    [提案手法の概要] について述べ, [具体的なアプローチ] を実装した.

    \item \textbf{実験による検証}:
    [実験設定の概要] においてシミュレーションを行い, [評価指標] を用いて性能を評価した.
    その結果, [主な結果の要約] ことを確認した.

    \item \textbf{特化と汎化の分析}:
    [特化パラメータと汎化パラメータの比較結果] から, [得られた知見・結論] を明らかにした.
\end{itemize}

以上の結果から, 本研究は [研究の意義・貢献] を示したといえる.

\subsection{今後の課題}
本研究に残された課題と, 将来の展望について述べる.

\begin{itemize}
    \item \textbf{[課題1のタイトル]}:
    [現状の限界] であるため, 今後は [解決策の方向性] を検討する必要がある.

    \item \textbf{[課題2のタイトル]}:
    本研究では [扱わなかった範囲] については考慮していないため, [拡張の可能性] が挙げられる.

    \item \textbf{[長期的展望]}:
    [より広い視点での課題] に対し, [将来的な応用可能性] を目指す.
\end{itemize}


% ------- 参考文献 -------
\bibliographystyle{kuisunsrt}
\bibliography{ref}

\subsection*{これから読む論文}
Mohamed S. Talamali, Genki Miyauchi, Thomas Watteyne, Micael S. Couceiro, Roderich Groß, "Ready, Bid, Go! On-Demand Delivery Using Fleets of Drones with
Unknown, Heterogeneous Energy Storage Constraints", AAMAS 2025, May 19 – 23, 2025, Detroit, Michigan, USA\\

Amila Thibbotuwawa, Grzegorz Bocewicz, Peter Nielsen, and Banaszak Zbigniew, "Planning deliveries with UAV routing under weather forecast and energy consumption constraints", IFAC PapersOnLine 52-13 (2019) 820–825\\

Giacomo Lodigiani, Nicola Basilico, Francesco Amigoni, "Robust Multi-Agent Pickup and Delivery with Delays", 
https://doi.org/10.48550/arXiv.


\end{document}




